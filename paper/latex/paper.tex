\documentclass[journal=jacsat,manuscript=article]{achemso}
\usepackage[parfill]{parskip}
\usepackage[version=3]{mhchem} % Formula subscripts using \ce{}
\usepackage[section]{placeins}
\newcommand*\mycommand[1]{\texttt{\emph{#1}}}
\title{surfinpy: A Surface Phase Diagram Generator}
\author{Adam R. Symington}
\email{A.R.Symington@bath.ac.uk}
\affiliation{Department of Chemistry, University of Bath, Claverton Down, Bath BA2 7AY, UK}
\author{Joshua Tse}
\affiliation{Department of Chemistry, University of Huddersfield, Queensgate, Huddersfield HD1 3DH, UK}
\author{Marco Molinari}
\affiliation{Department of Chemistry, University of Huddersfield, Queensgate, Huddersfield HD1 3DH, UK}
\author{Stephen C. Parker}
\email{S.C.Parker@bath.ac.uk}
\affiliation{Department of Chemistry, University of Bath, Claverton Down, Bath BA2 7AY, UK}

\date{2018\\ December}

\begin{document}


\section{Summary}
A surface phase diagram is a graphical representation of the different physical states of a surface under different conditions. 
The surface represents the first point of contact between the material and the environment. 
Thus understanding the state of surface is crucial for a wide range of problems in materials science concerning the relationship between 
the state of the surface and the surrounding environmental condtions. 
Examples include particle morphologies in solid state batteries \cite{Canepa2018}; 
determining the concentration of adsorbed water at a surface depending on synthesis conditions \cite{Molinari2012} \cite{Tegner2017}; 
catalytic reactions \cite{Reuter2003}; or determing the effect of dopants and impurities on the surface stability.  

Computational modelling can be used to generate surface phase diagrams from density functional theory (DFT) data.
One common research question is how water adsorption affects the surface of a material. 
To answer this a series of DFT calculations can be performed with varying concentrations of water on several different surfaces slabs. 
From this data the surface energy of each calculation (phase) as a function of temperature and pressure can be calculated according to the method employed in \cite{Molinari2012}. 
This data can be used across the entire dataset to determine which phase is most stable at a specific temperature and pressure and thus a phase diagram can be generated.

A further degree of complexity can be introduced by considering surface defects e.g. vacancies or interstitials, or other adsorbants e.g. carbon dioxide. 
Using surface defects as an example, it is important to consider the relationship between the defective surface, the stoichiometric surface and the adsorbing water molecules. 
A surface phase diagram can be contructed as a function of the chemcial potential of the adsorbing spcies (water) and the surface defect 
(e.g. oxygen if oxygen vacancies are being considered). This is done using the method of \cite{Marmier2004}. 

\section{surfinpy}

surfinpy is a Python module for generating a surface phase diagrams from DFT data. 
It contains two core modules for generating surface phase diagrams using both the methods employed in Molinari \textit{et al} and Marmier \textit{et al.}. 
These allow fast generation of temperature vs pressure phase diagrams and phase diagrams as a function of chemcial potential of species A and B. 
The plotting classes take the outputs of the calculation modules and generate phase diagrams using `matplotlib`. 
surfinpy is aimed towards theoretical solid state physicist and chemists who have a basic familiarity with Python. 
The repository contains examples of the core functionality as well as tutorials, implemented in jupyter notebooks to explain the full theory.
Furthermore, a detailed description of theory is also available within the documentation. 

\section{Acknowledgments}

ARS would like to thank Andrew R. McCluskey for his guidance through this project. This package was written during a PhD funded by AWE and EPSRC (EP/R010366/1). The input
data for the development and testing of this project was generated using ARCHER UK National Supercomputing Service (http://www.archer.ac.uk) via our membership of 
the UK's HEC Ma-terials Chemistry Consortium funded by EPSRC (EP/L000202).

\bibliography{paper}

\end{document}